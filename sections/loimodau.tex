\documentclass[../report.tex]{subfiles}
\begin{document}
\paragraph*{}Trong xã hội hiện đại, hệ thống kỹ thuật số đang trở nên phổ biến. Chúng ta dễ dàng nêu ra đươc các phát minh trong đó có sử dụng hệ thống số, đó là máy vi tính, điện thoại, các thiết bị di động, ...

\paragraph*{}Một hệ thống số bao gồm ba khối cơ bản: logic, bộ nhớ và giao tiếp. Khối logic có nhiệm vụ chuyển đổi và kết hợp dữ liệu. Khối bộ nhớ lưu trữ dữ liệu để truy xuất sau. Và khối giao tiếp có nhiệm vụ chuyển dữ liệu từ vị trí địa lý này tới vị trí khác.
Hiệu năng của hầu hết hệ thống số ngày nay bị giới hạn bởi khối giao tiếp của chúng. Cụ thể khối giao tiếp ở đây được thực hiện dưới mô hình mạng liên kết. Mạng liên kết có nhiệm vụ vận chuyển dữ liệu giữa các thành phần vật lý khác nhau của hệ thống số. Chình vì vậy để xây dựng được một hệ thống số hiệu quả, cần phải có sự hiểu biết về mạng liên kết. Trong tài liệu này, người viết xin phép được giới thiệu về các khái niệm chung trong mạng liên kết và việc triển khai mạng liên kết trong thực tế.
\paragraph*{}Em xin chân thành cảm ơn thầy và nhóm nghiên cứu đẫ hỗ trợ em hoàn thành tài liệu này.
\end{document}
